\documentclass[]{article}

%opening
\title{The guide to Pycle (\textbf{Py}thon \textbf{C}ompresive \textbf{Le}arning toolbox)}
\author{Vincent Schellekens}

\begin{document}

% TODO 
% - link to matlab toolbox

\maketitle

\begin{abstract}
	This is the guide to Pycle, a toolbox for Compressive Learning. It is structured as follows: first we shortly explain the theoretical methods this toolbox implements. Then, we explain how the toolbox is structured, and the main steps that a user should follow to use it. The detailed documentation of all the functionalities in the toolbox is then provided, followed by some practical examples to get started easily.
\end{abstract}


\section{What is Compressive Learning?}
% TODO

\section{An overview of Pycle}
% TODO explain how it is structured, how to use

% See papers for more

\section{Documentation}
\subsection{Sketching methods} 

\subsection{Learning tools} 

\subsection{Utilities} 

\section{Examples}
% TODO some examples, or put notebooks instead?

\end{document}
